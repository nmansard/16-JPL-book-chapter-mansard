\section{Planning acyclic contact sequences}
When planning the motion of a multiped robot, efficient
planners often rely on two simplifying assumptions:
firstly, the locomotion pattern is cyclic, and thus deterministic (when walking, after the left
foot the right foot always follows, and so on); secondly, the obstacles of the environment
can always be avoided in such a way that the first assumption is verified.

Discarding these assumptions is necessary for planning more complex motions.
For instance, for tasks such as standing up, climbing stairs
with a handrail, or getting out of a car, feasible solutions may require acyclic contact interactions between the robot and the environment.

However, acyclic contact planning requires 
handling a combinatorial, since any end effector can be used 
at any time, with any part of the environment. Moreover, the root path of the robot
must be computed at the same time as the contacts, adding a new layer of complexity~\citep{Bretl:2006:MPM:1124573.1124585, DBLP:conf/iser/EscandeKMG08}.

In this section, we describe a planner able to automatically compute 
a discrete sequence of contact configurations. These configurations verify a static equilibrium criterion,
and are used as an input guide path for the second step of our framework (Section TODO).

Contrary to previous contributions, the planner allows for \textit{interactive} performances.
This means that the next contact transition can be computed before the current contact transition
is executed on the robot. A pragmatic approach is chosen to break the combinatorial:
the planning of a root path is decoupled from the planning of the contacts.
First the planning of the root path is performed in a low dimensional space, which approximates 
the space of static equilibrium contact configurations (Section~\ref{sec:steve_root}).
Then a sequential algorithm is used to generate contacts along the computed path for the complete robot (Section~\ref{sec:steve_contact}).
As discussed in Section TODO decomposition results in a planner that can fail to compute a solution even if it exists.
However, experiments show that solutions can be effectively computed for various scenarios and robots, within seconds, where
complete systems can require up to hours of computation.

\subsection{Planning a root path}
\label{sec:steve_root}
In theory, decoupling the generation of a root path and contact configurations allows a strong 
reduction on the complexity of the acyclic contact planning problem, but raises a new scientific question: 
How to make sure that a root configuration allows to generate a contact configuration in static equilibrium?
We call such a root configuration an \textit{equilibrium feasible} configuration.
\cite{Bouyarmane2009} verify \textit{equilibrium feasibility} explicitly within their sampling based approach.
Because sampling a contact configuration is impossible (the contact manifold has a zero measure), when a root configuration is generated, a generalized inverse kinematics solver 
projects the robot onto a static equilibrium contact configuration. This is an expensive process, which prevents \textit{interactive} performances.

To avoid this explicit verification, we propose an inexpensive \textit{reachability condition}, drawn from an informal observation:
to create a contact with a surface, the root of the robot must be ``close, but not too close'': close enough
to allow contact creation; not too close to avoid colliding with the environment.
We then consider a dual representation to formulate this observation:
contact surfaces must lie in the reachable workspace of the robot limbs (Fig. TODO), and a scaling of the robot trunk must be collision free (Fig. TODO).
Given a parametrization, the condition defined by this representation is a trade-off between a necessary and a sufficient condition for the space of 
\textit{contact feasibile} root configurations, that is the space of root configurations which can lead to a contact.

To account for \textit{equilibrium feasibility} using the dual representation, we restrict our application domain.
We consider \textit{cluttered} problems, that admit as a solution a sequence of contact configurations for which at least one contact occurs with a surface whose friction cone contains the direction of the gravity (this includes all of the scenarios we mentionned).
For \textit{cluttered} problems, we assume \textit{equilibrium feasibility} is equivalent to \textit{contact feasibility}\footnote{We verify empirically this assumption in \cite{tonneauijrr16}}.

Under these assumptions, verifying that a root configuration is \textit{equilibrium feasible} requires a simple collision test thanks to the \textit{reachability condition}.
Therefore, to plan a root path for the robot, we can use any motion planner available from the litterature, such 
as the Bi-RRT planner \citep{770022}. With such a planner, restricting the configuration space of the robot to the low dimensional \textit{equilibrium feasible} space with the \textit{reachability condition} is trivial. We call our implementation of such a constrained planner RB-RRT, for \textit{Reachability-Based} RRT.

As an output of this phase, we obtain a continuous guide path of root configurations, assumed to be \textit{equilibrium feasible}.

\subsection{Generating contact configurations}
\label{sec:steve_contact}
The objective of this second phase is to extend a guide path for the root into a discrete sequence of full body, static equilibrium
contact configurations.

To achieve this, the input path is discretized, resulting in a list of root configurations.
The list is then traversed by the algorithm in an iterative fashion, starting from an input static equilibrium 
initial configuration. 

At each new step, an inverse kinematics solver tries to maintain all the contacts that existed in the previous step.
Contact failure can occur because of collisions or joint limit violation, in which case the contacts are invalidated.
Then, considering one contact free end-effector, a contact generator tries to generate a new contact that results in 
a static equilibrium configuration. If this fails, a new attempt is made with the next end-effector available, and so on.
Despite contact repositioning mechanisms to handle contact generation failure, the described algorithm can fail 
even if a feasible contact sequence exists. Again, this results from the trade-off made between efficiency and completeness, and empirical
results justify this approach. These statements are developed in in \cite{tonneauijrr16}.

